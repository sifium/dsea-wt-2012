Als Beispiel für einen randomisierten D\&C Algorithmus soll hier Quicksort dienen.

Quicksort ist ein Replace Verfahren, welches den Vorteil hat, \,dass man keinen \,zusätzlichen Speicher zum Sortieren benötigt.

Das Verfahren von Quicksort lässt sich gut an einem Beispiel erklären.
Wir nehmen hierzu ein Feld von Zahlen.
Man versucht beim partionieren mehr Intelligenz in die Sache zu stecken.
Wir suchen uns dazu ein beliebiges Element (Pivot- Element -$P$-) heraus und bilden zwei Mengen von Elementen, eine Menge $X\le P$ und eine Menge $Y\ge P$. Wir bewegen nun das Pivot-Element ans Ende (tauschen) und durchlaufen das Feld mit dem Index $i$ und schreiben die Zahlen kleiner $P$ an den Anfang und die Zahlen Größer $P$ an das Ende.
dann Tauschen wir das Pivot-Element wieder zurück und haben nun zwei Mengen größer, kleiner $P$ im Feld.
\begin{figure}[H]
\begin{verbatim}
partition(int a[],int l,int r){
   int x=a[r];
   int i=l-1;
   for(int j=l;j<r;j++){
      if(a[j]<=x){
         i=i+1;
         swap(a,i,j);
      }
   }
   swap(a,i+1,r);
   return i+1;
}
quicksort(int a[], int l, int r){
   if(l>=r) return;
   int p=partition(a,l,r);
   quicksort(a,l,p-1);
   quicksort(a,p+1,r);
}
\end{verbatim}
\caption{Quicksort}
\end{figure}
\subsection{Laufzeitanalyse}
\textbf{Worst-case}\\
Pivotelement $X$ ist immer das kleinste / größte Element in der betrachteten Teilfolge.
\begin{align*}
T(1)&=0\\
T(n)&=T(n-1)+cn\\
&=T(n-2)+c(n-1+n)\\
&=T(n-3)+c(n-2+n-1+n)\\
&\vdots\\
&=c\sum_{i=1}^n=c\frac{n(n+1)}{2}\in \theta(n^2)\\
\end{align*}
Erinnerung:\\
$$\mathcal X =\mbox{Zufallsvariable}$$
$$E(\mathcal X)=\sum pr(\mathcal X= x_i)x_i$$
z.B. Erwartete Augenzahl bei fairem Würfel
$$E(\mathcal X)=\sum_{i=1}^6\frac{1}{6}i=\frac{1}{6}\frac{67}{2}=3.5$$
z.B. Wieviele Münzwürfe benötigt man bis zum ersten Mal "Kopf" erscheint?
\begin{align*}
E(\mathcal X)&=\sum_{i=1}^\infty \left(\frac{1}{2}\right)^ii\\
E(\mathcal X)&=\frac{1}{2}\sum_{i=1}^\infty\left(\frac{1}{2}\right)^ii=\frac{1}{2}\frac{1}{(1-\frac{1}{2})^2}=\frac{1}{2}\frac{1}{\frac{1}{4}}\\&=2
\end{align*}
\fbox{\parbox{\columnwidth}{\textbf{Nebenrechnung:}\small\begin{align*}\sum_{i=1}^\infty ip^{i-1}&=\frac{d}{dp}\sum_{i=0}^\infty p^i\\
&=\frac{d}{dp}\frac{1}{1-p}=\frac{1}{(1-p)^2}\\&|p|<1\end{align*}}}\\[.5em]
Weiter mit QS:\\
\begin{align*}
T(n) &= \mbox{Erwartungswert}\\
T(n)&=\sum_{i=1}^n\frac{1}{n}\left(T(i-1)+T(n-i)\right)+cn\\
&\in \theta((n+1)ln(n+1)=\theta(n\log n)\\
nT(n)&=\sum_{i=1}^nT(i-1)+\sum_{i=1}^nT(n-i) +cn^2\\
&=2\sum_{i=0}^{n-1}T(i)+cn^2\\
&(n-1)T(n-1)\\
&=2\sum_{i=0}^{n-2}T(i)+c(n-1)^2\end{align*}
\tiny\[nT(n)-(n-1)T(n-1)=2T(n-1)+c(n^2-(n^2-2n+1))\]
\small\begin{align*}
nT(n)&=(n+1)T(n-1)+c'n\\
T(n)&=\frac{n+1}{n}T(n-1)+c'\\
&=\frac{n+1}{n}\left(\frac{n}{n-1}T(n-2)+c'\right)+c'\\
&=\frac{n+1}{n-1}T(n-2)+\frac{n+1}{n}c'+\frac{n+1}{n+1}c'\\
&=\frac{n+1}{n-k}T(n-k-1)+c'(n+1)\sum_{i=n-k+1}^{n+1}\frac{1}{i}\\
&6k:=n-1\\
&=(n+1)T(0)+c'(n+1)\sum_{i=1}^{n+1}\frac{1}{i}
\end{align*}
\normalsize